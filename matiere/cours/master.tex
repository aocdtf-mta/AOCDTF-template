%--------------------------------------
%appel de la classe de document et de ses options
%--------------------------------------

\documentclass[a4paper, 11pt, twoside, fleqn]{memoir}

\usepackage{AOCDTF}

\addbibresource{../../bibliographies/exemple.bib}

%--------------------------------------
%données du document
%--------------------------------------

\formationtrue %si \formationfalse alors l'intitulé de la formation n'apparait pas sur la page de titre

\corpsdemetier{Corps de métier}
\siglemetier{choix}
\nombreauteur{2}

\formation{Formation}
\matiere{Matière}
\cours{Cours}

\auteura{Prénom}{Nom}
\siglemetierauteura{choix} 
\auteurb{Prénom}{Nom}
\siglemetierauteurb{choix}
\auteurc{Prénom}{Nom}
\siglemetierauteurc{choix}
\auteurd{Prénom}{Nom}
\siglemetierauteurd{choix}

\typemedia{paper} %choix screen ou paper pour les vidéos et schémas animés

\decoupagechapitre{5} %espacement des marqueurs entre les différents chapitres (à régler en fin de rédaction) (5cm vaut un espacement équivalement à la hauteur du marqueur, une page ne peut en contenir que 5 avec cet espacement-la mais il est le plus équilibré)

%--------------------------------------
%corps du document
%--------------------------------------

\begin{document} %corps du document

%--------------------------------------
%préface, page de couverture, table des matières...
%--------------------------------------

\frontmatter
	
\Framefalse %défini la booléenne Frame comme faux
	
	%--------------------------------------
	%page de couverture et de titre
	%--------------------------------------

\pagecouverture
\pagetitre
\marqueurchapitre

\media{\thetypemedia}{}{
	\makeatletter
	\setboolean{@twoside}{false}
	\makeatother}
	
	\pagestyle{plain} %style de page avec en-tête et pied-de-page
	\openany
	
	%--------------------------------------
	%listes de contenus
	%--------------------------------------
	
	{\hypersetup{linkcolor=black}\tableofcontents*} %table des matières en noir
	\newpage
	{\hypersetup{linkcolor=black}\listoftables*} %liste des tableaux en noir
	\newpage
	{\hypersetup{linkcolor=black}\listoffigures*} %liste des figures en noir
	{\hypersetup{linkcolor=black}\tcblistof[\chapter*]{frm}{Liste des formules}} %liste des équations en noir
	{\hypersetup{linkcolor=black}\tcblistof[\chapter*]{dfn}{Liste des définitions}} %liste des définitions en noir
	{\hypersetup{linkcolor=black}\tcblistof[\chapter*]{xmpl}{Liste des exemples}} %liste des définitions en noir

	\media{\thetypemedia}{\openright}{} %début de chapitre à "droite" mais comme demarrage de la numérotation inversé avec la page de titre, ça décale l'ouverture des chapitre à gauche

	%--------------------------------------
	%chapitre d'introduction
	%--------------------------------------

	\documentclass[a4paper, 11pt, twoside, fleqn]{memoir}

\usepackage{AOCDTF}

%--------------------------------------
%entrées du glossaire
%--------------------------------------

%création de macro-commande pour automatiser la rédaction de nouvelles entrées référencés dans le glossaire

\newglossaryentry{ex}{name={exemple}, description={définition de l'exemple d'entrée classique dans le glossaire}}
%--------------------------------------
%entrées des acronymes
%--------------------------------------

\newacronym{aocdtf}{AOCDTF}{Association Ouvrière des Compagnons du Devoir et du Tour de France}


\typemedia{paper} %choix screen ou paper pour les vidéos et schémas animés

\marqueurchapitre
\decoupagechapitre{1} %juste pour éviter les erreurs lors de la compilation des sous-programmations (passera en commentaire)

%lien d'édition des figures Tikz sur le site mathcha.io (rajouter le lien d'une modification effectuée sur la figure tikz avec le nom du modificateur car il n'y a qu'un lien par compte)

%lien mathcha Nom Prénom : 

%--------------------------------------
%corps du document
%--------------------------------------

\begin{document} %corps du document

	\chapter{Préface}

Ce \emph{template} contient un recueil neutre des programmations \emph{types} dont les programmeurs devront fortement s'inspirer pour aboutir à des documents unifiés graphiquement et à la présentation irréprochable.\\

Celui-ci se veut exhaustif quant aux nombreux cas de figures auxquels les programmeurs feront face, comme l'insertion de figures avec toutes options que cela comporte, ou encore le dessin sous \LaTeX{}... La liste est longue et je renvoie vers la table des matière pour avoir une vue d'ensemble face aux différentes programmations types. La liste des exemples donne également une navigation rapide parmi les différents exemples de codes.\\ 
Il s'agit la d'un outil permettant de faciliter grandement la programmation de nouveaux documents, à conserver en toile de fond lors de la prise en main de \LaTeX{}\ldots\\

L'outil \LaTeX{} étant bien conçu (et bien amélioré par le package AOCDTF), toutes les références listées ainsi que toutes les divisions ou encore l'entièreté de la table des matière sont référencées -- bien que ça ne soit mis en évidence par des couleurs pour éviter un arc-en-ciel illisible -- et il suffit de cliquer sur ces éléments pour être renvoyé directement sur leur localisation dans le document. Il s'agit la d'une des nombreuses valeurs ajoutées de \LaTeX{}.



\end{document}
		
%--------------------------------------
%corps de texte, annexes
%--------------------------------------

\mainmatter

\Frametrue %défini la booléenne Frame comme vrai -> marqueurs de chapitre

	%--------------------------------------
	%inclusion des chapitres
	%--------------------------------------

\input{chap_premier_chapitre}

\input{chap_deuxieme_chapitre}

\input{chap_troisieme_chapitre}

	%--------------------------------------
	%style des annexes
	%--------------------------------------

	\Framefalse %défini la booléenne Frame comme false -> pas de marqueurs de chapitre
	\appendix %appel des annexes
	\appendixpage

	%--------------------------------------
	%inclusion des chapitres
	%--------------------------------------

	\input{ann_premiere_annexe}
	
	\input{ann_deuxieme_annexe}

%--------------------------------------
%conclusion, bibliographie
%--------------------------------------

\backmatter

	%--------------------------------------
	%inclusion des chapitres
	%--------------------------------------

	\input{chap_conclusion}

	%--------------------------------------
	%bibliographie
	%--------------------------------------

	\nocite{AuteurAnnee, Organisme:Numeronorme-Annee, Site:abreviation}
	
	\printbibliography %ajout des références bibliographiques
		
\end{document}

